\documentclass{../upb-cn}

\usepackage{pgfgantt}

\lheader{Seminar: XXX (SS24)}  % XXX is the name of the seminar
\rheader{Your Name}

\addbibresource{refs.bib}

\title{Seminar: XXX (SS24)}
\author{Your Anonymous Friend}

\begin{document}
\makethetitle

\begin{notebox}
    \begin{tabular}{@{}ll}
        \textbf{Research group:} & Computer Networks (CN) \\
        \textbf{Study program:} & BSc/MSc Computer Science / Computer Engineering \\
        \textbf{Supervisor:} & Prof. Dr. Lin Wang \\
        \textbf{Paper title:} & Title of the reviewed report
    \end{tabular}
\end{notebox}

% Please remove the following part in your report
The review should be 1--2 pages long in total. Reviews that are too short are unlikely to be helpful and will be penalized. The following structure should be followed strictly in your review.

\section{Summary of the Paper}

What do you think the paper is about? What contributions has the paper made? Is the paper technically sound? Is the studied problem significant? Is the idea novel? Is the writing quality appropriate?


\section{Strengths}

State in a bullet-point fashion what is good about this paper. Examples:

\begin{enumerate}
    \item The paper studies an interesting problem that has been overlooked in the networking community so far.
    \item The paper conducts thorough experimental studies to verify its claims. 
\end{enumerate}

\section{Areas for Improvement}

State in a bullet-point fashion what is not so good about this paper. Examples:

\begin{enumerate}
    \item The paper makes a core assumption... that may not hold in real-world scenarios.
    \item The writing of the paper needs improvement; there are many typos.
\end{enumerate}

\section{Detailed Comments}

Now, please write down your detailed comments: For each of the above bullet points, justify it and explain in detail your reasoning behind it. Then, provide (actionable) suggestions for the authors to improve the paper. Respect the authors and try to be polite, constructive, less assertive, and emotionally mature in your comments. 


\end{document}