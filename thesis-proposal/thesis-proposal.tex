\documentclass[report]{../upb-cn}

\usepackage{pgfgantt}

\lheader{BSc/MSc Thesis}
\rheader{Your Name}

\addbibresource{refs.bib}

\title{Thesis Proposal: Tentative Title of Your Thesis}
\author{Your Name (Matriculation Number)}

\begin{document}
\maketitle

\begin{notebox}
    \begin{tabular}{@{}ll}
        \textbf{Research group:} & Computer Networks (CN) \\
        \textbf{Study program:} & BSc/MSc Computer Science / Computer Engineering \\
        \textbf{First reviewer:} & Prof. Dr. Lin Wang \\
        \textbf{Second reviewer:} & Someone at the department with a PhD \\
        \textbf{Daily supervisor(s):} & Person who talk to you on a daily/weekly basis
    \end{tabular}
\end{notebox}

\section{Introduction}
\label{sec:introduction}

Provide the background for your thesis project and motivate your proposed work. At the end of this section, provide a short summary of your expected contributions. You can consider the following questions when writing:

\begin{itemize}
    \item What problem are you going to work on?
    \item Why is it an important problem?
    \item Why does the problem have not been solved already?
    \item What new insights and ideas do you have?
    \item What is the overarching research question of your thesis work?
    \item What contributions do you expect to make? 
\end{itemize}

\section{Research Objectives}
\label{sec:objectives}

Provide more details about the overarching research question, set up a high-level goal, and divide the goal into multiple concrete objectives. Provide details for each of these objectives. Try to limit the number of objectives in the range of two to four. 

\section{Related Work}
\label{sec:relatedwork}

Describe what has been done by others. What are the important references? Among these which ones are your direct competitors and which ones form the base of your own work? Explain in detail. For the direct competitors explain clearly what is still lacking from them to address your research question. 

\section{Proposed Work}
\label{sec:proposedwork}

Describe the proposed work you will perform during your thesis. To achieve each of the objectives listed in Section~\ref{sec:objectives}, what general approaches will you take? What techniques will be used? What innovative ideas will you explore? How are you going to evaluate your proposed ideas? It would be even better if you could already include some preliminary investigation results (e.g., a simplified case study, some measurement results) to support your arguments. That would significantly increase the credibility of your proposal.

\section{Timeline}
\label{sec:timeline}

The following Gantt chart is required. Change the titles with ``Month'' with your actual month information. In the chart, we have already listed the most important entries for a typical systems-oriented thesis, but you can add or remove entries if necessary. Adjust the duration of each entry according to your thesis project. You may also provide a more detailed list of milestones to achieve along the project execution.

\begin{figure}[!h]
\begin{ganttchart}[
    vgrid={*{3}{gray, dotted}, *1{black, dashed}},
    bar label node/.append style={
        align=right,
        text width=0.15\textwidth}
    ]{1}{24}
    \gantttitle{Month 1}{4} \gantttitle{Month 2}{4}  \gantttitle{Month 3}{4} \gantttitle{Month 4}{4} \gantttitle{Month 5}{4} \gantttitle{Month 6}{4}\\
    \ganttbar{Literature \\ study}{1}{3} \\
    \ganttbar{Design}{4}{10} \\
    \ganttbar{Implementation}{8}{15} \\
    \ganttbar{Evaluation}{12}{18} \\
    \ganttbar{Writing \& \\ presentation}{17}{20}
\end{ganttchart}
\end{figure}

\begin{table}[!h]
    \centering
    \begin{tabular}{@{}rl@{}}
        \toprule
        \multicolumn{2}{@{}l}{\textbf{Milestones}} \\
        \midrule
        Date A & Finish the literature study and generate a comparative table \\
        Date B & Complete the sketch of the system design \\
        Date C & Start implementation of the system \\
        Date D & Complete the system setup and start evaluation \\
        ... & ... \\
        \bottomrule
    \end{tabular}
\end{table}

\printbibliography

\newpage

% remove this tutorial part in your report
\section{How to Use This Template for Writing}

% \subsection{Subsection Heading}
% \subsubsection{Subsubsection Heading (Avoid Using It If Possible)}

By default, paragraphs are not indented. To cite a reference, you can use the \verb+\cite+ command. Here is an example: HIRE is a novel resource scheduler for in-network computing~\cite{2021:asplos:hire}. The list of references is shown at the end of the document in the ``References'' section. We use \verb+biblatex+ to manage references and the source of bib items is specified at the beginning with the command \verb+\addbibresource{...}+. It is recommended that you collect the bib entries from DBLP\footnote{DBLP: https://dblp.org}.

You can also create unnumbered (and numbered) lists as in the following examples. Note that the list should not go deeper than two levels; otherwise, it becomes ugly. 

\begin{itemize}
    \item First item
    \item Second item 
        \begin{itemize}
            \item First subitem
            \item Second subitem
        \end{itemize}
\end{itemize}

If you have some text you want to put in monospace (e.g., cite something in verbatim), you can use the \verb+\verb+ command to do that. Alternatively, you can use \verb+\lstinline[language=...]|...|+ with syntax highlights if you specify the language (e.g., Python, C, or C++). If you want to write a code block, you can use the \verb+lstlisting+ environment with syntax highlights. Here is an example for a C code snippet.

\begin{lstlisting}[language=C]
// Simple C example
int main(int argc, char** argv) {
    return 0;
}
\end{lstlisting}

Figures and tables should generally be put at the top of the page if not on the title page. Table~\ref{tab:info} is an example for the table format. Avoid using vertical bars in a table unless it is really necessary. All cells should be left-aligned except cells with numbers which should be right-aligned or dot-aligned. The caption for the table should sit at the top of the table, while it is at the bottom for figures.

\begin{table}[!ht]
    \centering
    \caption{Course Grade}\label{tab:info}
    \begin{tabular}{@{}lrr@{}}
        \toprule
        \textbf{Name} & \textbf{Matriculation Number} & \textbf{Grade} \\
        \midrule
        Max Mustermann & 112233 & 1.3 \\
        Paul M\"uller & 445566 & 1.7 \\
        \bottomrule
    \end{tabular}
\end{table}

\end{document}